\documentclass[12pt]{article}

\usepackage{EngReport}

\graphicspath{{Images/}}
\bibliography{Sources}
\onehalfspacing
\graphicspath{{images/}}
\geometry{letterpaper, portrait, includeheadfoot=true, hmargin=1in, vmargin=1in}

\renewcommand\thesection{(\arabic{section})}
\renewcommand\thesubsection{(\arabic{section}.\arabic{subsection})}
\renewcommand{\theequation}{\arabic{section}.\arabic{equation}}
\everymath{\displaystyle}

%\fontsize{font size}{vertsize (usually 1.2x)}\selectfont

\begin{document}
\renewcommand{\familydefault}{\rmdefault}

\begin{titlepage}
    \begin{center}
    {\fontsize{40}{48}\selectfont \bfseries Homework 6} 
    \\\vspace{20pt}
    {\LARGE PHYS209} \\
    \vspace{20pt}
    \textbf{Hikmat Gulaliyev}
    \vspace{8pt}
    \\ 17.11.2023
    \end{center}
\end{titlepage}
\pagestyle{fancy}
\fancyhf{}
\setlength{\headheight}{30pt}
\renewcommand{\headrulewidth}{0.4pt}
\renewcommand{\footrulewidth}{0.4pt}
\lhead{\large Homework 10}
\rhead{\large PHYS209 \\ Hikmat Gulaliyev}
\rfoot{\textbf{Page \thepage}}
\lfoot{}

\section{Problem 1}
\subsection{}
Since $\mathcal{D}\cdot f(x)$ is equal to:
\begin{equation}
    \label {eq:fx}
    \left(\frac{\mathrm{d}^3}{\mathrm{d}x^3}+\cos(x) \frac{\mathrm{d}^2}{\mathrm{d}x^2}+x^3 \frac{\mathrm{d}}{\mathrm{d}x} + 1\right)f(x) = 0
\end{equation} 
And since $\mathcal{A}$ is equal to \ref{eq:fx}, we can rewrite it as matrix as:
\begin{equation}
    \frac{\mathrm{d}}{\mathrm{d}x} \begin{pmatrix}
        f(x) \\
        f'(x) \\
        f''(x) \\
    \end{pmatrix} = \begin{pmatrix}
        0 & 1 & 0 \\
        0 & 0 & 1 \\
        -1 & -\cos(x) & -x^3 \\
    \end{pmatrix} \begin{pmatrix}
        f(x) \\
        f'(x) \\
        f''(x) \\
    \end{pmatrix}
\end{equation}
\section{Problem 2}
\subsection{}
Starting with matrix form of equation:
\begin{equation}
    \frac{\mathrm{d}}{\mathrm{d}x}
    \begin{pmatrix}
        f(x) \\
        f'(x) \\
        \vdots \\
        f^{(n-2)}(x) \\
        f^{(n-1)}(x) \\
    \end{pmatrix} = \begin{pmatrix}
        0 & 1 & 0 & \cdots  & 0 \\
        0 & 0 & 1 & \cdots  & 0 \\
        \vdots & \vdots & \vdots & \ddots & \vdots \\
        0 & 0 & 0 & \cdots & 1\\
        -a_n & -a_{n-1} & -a_{n-2} & \cdots & a_1\\ 
    \end{pmatrix}
    \begin{pmatrix}
        f(x) \\
        f'(x) \\
        \vdots \\
        f^{(n-2)}(x) \\
        f^{(n-1)}(x) \\
    \end{pmatrix}
\end{equation}
We can apply differential operator, and multiply matrices to get:
\begin{equation}
    \begin{pmatrix}
        f'(x) \\
        f''(x) \\
        \vdots \\
        f^{(n-1)}(x) \\
        f^{(n)}(x) \\
    \end{pmatrix}  = 
    \begin{pmatrix}
        f'(x) \\
        f''(x) \\
        \vdots \\
        f^{(n-1)}(x) \\
        -a_n f(x) - a_{n-1} f'(x) - \cdots - a_1 f^{(n-1)}(x) \\
    \end{pmatrix}
\end{equation}
And last line gives us:
\begin{equation}
    f^{(n)}(x) = -a_n f(x) - a_{n-1} f'(x) - \cdots - a_1 f^{(n-1)}(x)
\end{equation}
Which is the same as:
\begin{equation}
    f^{(n)}(x) + a_1 f^{(n-1)}(x) + \cdots + a_n f(x) = 0
\end{equation}
\subsection{}
We are given that:
\begin{equation}
\begin{gathered}
    \det_n(\mathcal{M}) = \sum_{i_1,...,i_n}\epsilon_{i_1,...,i_n}\mathcal{M}_{1i_1}\mathcal{M}_{2i_2}\cdots\mathcal{M}_{ni_n} = \\
    -\lambda\sum_{i_2,...,i_n}\epsilon_{1,i_2,...,i_n}\mathcal{M}_{2i_2}\mathcal{M}_{3i_3}\cdots\mathcal{M}_{ni_n}+
    \sum_{i_2,...,i_{n}}\epsilon_{2,i_2,...,i_{n}}\mathcal{M}_{2i_2}\mathcal{M}_{3i_3}\cdots\mathcal{M}_{ni_n}
\end{gathered}
\end{equation}
We need to prove that the second term can be written as:
\begin{equation}
    \begin{gathered}
    \sum_{i_2,...,i_{n}}\epsilon_{2,i_2,...,i_{n}}\mathcal{M}_{2i_2}\mathcal{M}_{3i_3}\cdots\mathcal{M}_{ni_n} 
     =  -a_n\sum_{i_2,...,i_{n-1}}\epsilon_{2,i_2,...,i_{n-1},1}\mathcal{M}_{2i_1}\mathcal{M}_{3i_3}\cdots\mathcal{M}_{n-1,i_{n-1}}
    \end{gathered}
\end{equation}
First, if we choose $i_n$ to be equal to 1, it is obvious that we will get the out equation to the given form. However, if we choose $i_n$ to be anything but 1, then we will run into problems. Such as if we choose $i_n$ to be equal to 2, the value of $\epsilon$ will always be equal to zero, and for any $i_n > 2$ we will have our $\mathcal{M}_{2i_2}\mathcal{M}_{3i_3}\cdots\mathcal{M}_{ni_n}$ to be 0, since $\mathcal{M}_{3,i_3}$ will take zero for anything but 3 and in that case $\epsilon$ will be zero.
\subsection{}
To further simplify the answer of the previous part, we can look at different values that $i_2$ can take. It is obvious that it can't take 1 or 2 since that would make $\epsilon$ to be equal to zero. If we take $i_2$ to be equal to 3 than our equation would take the form:
\begin{equation}
    \begin{gathered}
        -a_n\sum_{i_3,...,i_{n-1}}\epsilon_{2,3,i_4,...,i_{n-1},1}\mathcal{M}_{3i_3}\mathcal{M}_{4i_4}\cdots\mathcal{M}_{n-1,i_{n-1}}
    \end{gathered}
\end{equation}
And since $\mathcal{M}_{2,n}$ is 0 for any value bigger than 3, $\mathcal{M}_{2i_2}\mathcal{M}_{3i_3}\cdots\mathcal{M}_{ni_n}$ will be equal to 0, therefore $i_2 = 3$
Repeating same process for every $i_n$ form 3 up to n we will get:
\begin{equation}
    -a_n\epsilon_{2,3,4...,n-1,n,1}
\end{equation}
\subsection{}
Lets look at the case n = 3:
\begin{equation}
    \begin{gathered}
        -a_n\epsilon_{2,3,1} = a_n\epsilon_{2,1,3} = -a_n\epsilon_{1,2,3} = -a_n 
    \end{gathered}
\end{equation}
So parity is 2, similarly for n = 4:
\begin{equation}
    \begin{gathered}
        -a_n\epsilon_{2,3,4,1} = a_n\epsilon_{2,3,1,4} = -a_n\epsilon_{2,1,3,4} = a_n\epsilon_{1,2,3,4} = a_n
    \end{gathered}
\end{equation} 
So parity is 3, and so by induction, we can say that parity is n-1. Therefore:
\begin{equation}
    \epsilon_{2,3,4,...,n,1}  = (-1)^{n-1}
\end{equation}
Therefore, determinant is equal to:
\begin{equation}
    det_n(\mathcal{M}) = -\lambda\sum_{i_2,...,i_n}\epsilon_{1,i_2,...,i_n}\mathcal{M}_{2i_2}\mathcal{M}_{3i_3}\cdots\mathcal{M}_{ni_n} + (-1)^{n-1}a_n
\end{equation}
\subsection{}
Since the properties of permutations, adding 1 to the beginning of permutation does not change the parity, as parity is number of swaps required to return it to original ordering. And since $\epsilon_{1,i_2,...,i_n}$ represents permutation of set $\{1,2,3,...,n\}$, and $\epsilon_{k_1,k_2,...,k_{n-1}}$ is of set $\{2,3,...,n\}$
therefore parities for two permutations are equal and, $\epsilon_{1,i_2,...,i_n} = \epsilon_{k_1,k_2,...,k_{n-1}}$
\subsection{}
Starting with $n=1$:
\begin{equation}
    det_1(\mathcal{M}) = -\lambda - a_1
\end{equation}
If $n=2$:
\begin{equation}
\begin{gathered}
    det_2(\mathcal{M}) = -\lambda det_1(\mathcal{M}) + a_2 \\=-\lambda(-\lambda - a_1) + a2 \\ = \lambda^2 + \lambda a_1 + a_2
\end{gathered}
\end{equation}
For $n=3$:
\begin{equation}
\begin{gathered}
    det_3(\mathcal{M}) = -\lambda det_2(\mathcal{M}) + (-1)^{3-1}a_3 \\ = -\lambda(\lambda^2 + \lambda a_1 + a_2) + a_3 \\ = -\lambda^3 - \lambda^2 a_1 - \lambda a_2 + a_3
\end{gathered}
\end{equation}
\end{document}

