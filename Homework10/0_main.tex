\documentclass[12pt]{article}

\usepackage{EngReport}

\graphicspath{{Images/}}
\bibliography{Sources}
\onehalfspacing
\graphicspath{{images/}}
\geometry{letterpaper, portrait, includeheadfoot=true, hmargin=1in, vmargin=1in}

\renewcommand\thesection{(\arabic{section})}
\renewcommand\thesubsection{(\arabic{section}.\arabic{subsection})}
\renewcommand{\theequation}{\arabic{section}.\arabic{equation}}
\everymath{\displaystyle}

%\fontsize{font size}{vertsize (usually 1.2x)}\selectfont

\begin{document}
\renewcommand{\familydefault}{\rmdefault}

\begin{titlepage}
    \begin{center}
    {\fontsize{40}{48}\selectfont \bfseries Homework 6} 
    \\\vspace{20pt}
    {\LARGE PHYS209} \\
    \vspace{20pt}
    \textbf{Hikmat Gulaliyev}
    \vspace{8pt}
    \\ 17.11.2023
    \end{center}
\end{titlepage}
\pagestyle{fancy}
\fancyhf{}
\setlength{\headheight}{30pt}
\renewcommand{\headrulewidth}{0.4pt}
\renewcommand{\footrulewidth}{0.4pt}
\lhead{\large Homework 10}
\rhead{\large PHYS209 \\ Hikmat Gulaliyev}
\rfoot{\textbf{Page \thepage}}
\lfoot{}
\section{Problem 1}
\subsection{}
If we declare $z = e^{i\pi\theta}$, then $z^* = e^{-i\pi\theta}$. Therefore $f(z^*)$:
\begin{equation}
    \begin{gathered}
        \sin(1) + \cos(1) = 1.38177329068\\
        \sin(e^{-i\pi/4}) + \cos(e^{-i\pi/4}) = 0.76536686473\\
        \sin(e^{-i\pi/2}) + \cos(e^{-i\pi/2}) =  0.38177329068\\
        \sin(e^{-3i\pi/4}) + \cos(e^{-3i\pi/4}) = 0.76536686473\\
    \end{gathered}
\end{equation}
\subsection{}
From Euler's formula we know that $e^{i\theta} = \cos(\theta) + i\sin(\theta)$. From which we can derive:
\begin{equation}
    \label {eq:cos}
    \begin{gathered}
        \cos(\theta) = \frac{e^{i\theta} + e^{-i\theta}}{2}\\
        \sin(\theta) = \frac{e^{i\theta} - e^{-i\theta}}{2i}
    \end{gathered}
\end{equation}
Putting these into $f(z)$ we get:
\begin{equation}
    f(z) = \frac{e^{iz}-e^{-iz}}{2i} + \frac{e^{iz}+e^{-iz}}{2}
\end{equation}
Getting complex conjugate of $f(z)$:
\begin{equation}
    \begin{split}
        (f(z))^* &= \left(\frac{e^{iz}-e^{-iz}}{2i} + \frac{e^{iz}+e^{-iz}}{2}\right)^*\\
        &= \left(\frac{e^{iz}-e^{-iz}}{2i}\right)^* + \left(\frac{e^{iz}+e^{-iz}}{2}\right)^*\\
        &= \frac{e^{-iz^*}-e^{iz^*}}{-2i} + \frac{e^{-iz^*}+e^{iz^*}}{2}\\
        &= \frac{e^{iz^*}-e^{-iz^*}}{2i} + \frac{e^{-iz^*}+e^{iz^*}}{2}\\
        &= f(z^*)
    \end{split}
\end{equation} 
\subsection{}
If we define $g(z)$ to be equal $\cos(iz)$ then using \ref{eq:cos} we get:
\begin{equation}
        g(z) = \cos(iz) = \frac{e^{-z} + e^z}{2}
\end{equation}
Then to find complex conjugate of $g(z)$ we get:
\begin{equation}
    \begin{split}
        (g(z))^* &= \left(\frac{e^{-z} + e^z}{2}\right)^*\\
        &= \frac{e^{-z^*} + e^{z^*}}{2}\\
        &= g(z^*)
    \end{split}
\end{equation}
Similarly if we define $h(z)$ to be equal $\sin(iz)$ then using \ref{eq:cos} we get:
\begin{equation}
        h(z) = \sin(iz) = \frac{e^{-z} - e^z}{2i}
\end{equation}
Then to find complex conjugate of $h(z)$ we get:
\begin{equation}
    \begin{split}
        (h(z))^* &= \left(\frac{e^{-z} - e^z}{2i}\right)^*\\
        &= \frac{e^{-z^*} - e^{z^*}}{-2i}\\
        &= -h(z^*)\\
        &\neq h(z^*)
    \end{split}
\end{equation}
\section{Problem 2}
\subsection{}
To derive most general $3x3$ Hermitian matrix, we can start with a general $3x3$ matrix:
\begin{equation}
    \begin{pmatrix}
        a + ia' & b + ib'  & c + ic'\\
        d + id' & e + ie' & f + if'\\
        g + ig' & h + ih' & k + ki'
    \end{pmatrix}
\end{equation}
Where $z$ is real part $z_{ij}$ and $z'$ is imaginary part. Now finding complex conjugate of the matrix:
\begin{equation}
    \begin{pmatrix}
        a - ia' & d - id'  & g - ig'\\
        b - ib' & e - ie' & h - ih'\\
        c - ic' & f - if' & k - ki'
    \end{pmatrix}
\end{equation}
Since matrix is Hermitian, it must be equal to its complex conjugate. Therefore:
\begin{equation}
    \begin{gathered}
        a + ia' = a - ia' \Rightarrow a' = 0\\
        b + ib' = d - id' \Rightarrow b' = -d', b = d\\
        c + ic' = g - ig' \Rightarrow c' = -g', c = d\\
        e + ie' = e - ie' \Rightarrow e' = 0\\
        f + if' = h - ih' \Rightarrow f' = -h', f = h\\
        k + ki' = k - ki' \Rightarrow k' = 0
    \end{gathered}
\end{equation} 
Therefore most general $3x3$ Hermitian matrix is:
\begin{equation}
    \begin{pmatrix}
        a & b - ib'  & c - ic'\\
        b + ib' & e & f - if'\\
        c + ic' & f + if' & k
    \end{pmatrix}
\end{equation}
\end{document}
