\documentclass[12pt]{article}

\usepackage{EngReport}

\graphicspath{{Images/}}
\bibliography{Sources}
\onehalfspacing
\graphicspath{{images/}}
\geometry{letterpaper, portrait, includeheadfoot=true, hmargin=1in, vmargin=1in}

\renewcommand\thesection{(\arabic{section})}
\renewcommand\thesubsection{(\arabic{section}.\arabic{subsection})}
\renewcommand{\theequation}{\arabic{section}.\arabic{equation}}

%\fontsize{font size}{vertsize (usually 1.2x)}\selectfont

\begin{document}
\renewcommand{\familydefault}{\rmdefault}

\begin{titlepage}
    \begin{center}
    {\fontsize{40}{48}\selectfont \bfseries Homework 6} 
    \\\vspace{20pt}
    {\LARGE PHYS209} \\
    \vspace{20pt}
    \textbf{Hikmat Gulaliyev}
    \vspace{8pt}
    \\ 17.11.2023
    \end{center}
\end{titlepage}
\pagestyle{fancy}
\fancyhf{}
\setlength{\headheight}{30pt}
\renewcommand{\headrulewidth}{0.4pt}
\renewcommand{\footrulewidth}{0.4pt}
\lhead{\large Homework 10}
\rhead{\large PHYS209 \\ Hikmat Gulaliyev}
\rfoot{\textbf{Page \thepage}}
\lfoot{}

\section{Problem One}
\subsection{}
The higher order function \textit{sqr2} can be defined as:
\begin{equation}
    sqr::(\mathbb{C}\rightarrow\mathbb{C}) \rightarrow (\mathbb{C}\rightarrow\mathbb{C})
\end{equation}
\begin{equation}
    sqr2 = (x \rightarrow f(x)) \rightarrow (x \rightarrow 2f(x)) 
\end{equation}

\subsection{}
Since $\cos(x)' = -\sin(x)$ the type and definition of high order function $(\frac{d}{dx} + I)$:
\begin{equation}
    \left(\frac{d}{dx} + \mathcal{I}\right)\cdot\cos(x)::(\mathbb{C}\rightarrow\mathbb{C}) \rightarrow (\mathbb{C}\rightarrow\mathbb{C})
\end{equation}
\begin{equation}
    \left(\frac{d}{dx} + \mathcal{I}\right)\cdot\cos(x) = (x \rightarrow \cos(x)) \rightarrow (x \rightarrow \cos(x) - \sin(x))
\end{equation}

\subsection{}
Type of operator \textit{C} when acting on real variables and real functions is:
\begin{equation}
    C::[(\mathbb{R}\rightarrow\mathbb{R})\rightarrow(\mathbb{R}\rightarrow\mathbb{R})] \rightarrow[(\mathbb{R}\rightarrow\mathbb{R})\rightarrow(\mathbb{R}\rightarrow\mathbb{R})]
\end{equation}

\subsection{}
Action of $\exp\left(\frac{d}{dx}\right)x^4$  can be calculated as:
\begin{multline}
    \exp\left(\frac{d}{dx}\right)x^4 = (x^4)^{(0)} + \frac{1}{1!}(x^4)^{(1)} + \frac{1}{2!}(x^4)^{(2)} + \frac{1}{3!}(x^4)^{(3)} + \frac{1}{4!}(x^4)^{(4)} + ...  =\\x^4 + 4x^3 + 6x^2 + 4x + 1
\end{multline}
Since $(x^4)^{(n)}$ is equal to 0 for any $n>4$ it is easy to calculate analytically
\subsection*{}
However, if we simplify (6) even further it can be seen that
\begin{equation}
    4x^3 + 6x^2 + 4x + 1 = (x+1)^4
\end{equation}
Which amazingly coincides with its geometrical meaning of shifting argument
\end{document}
